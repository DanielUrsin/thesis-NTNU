\chapter{Introduction}
\section{The importance of information}
Access to reliable and timely information has always been the backbone of effective emergency response and law enforcement operations. The right information at the right time could ensure successful handling of dangerous situations while simultaneously minimizing the risk for all parties involved. Situational awareness was conceptualized by Endsley \cite{endsley_endsley_1995} as \textit{“the perception of the elements in the environment within a volume of time and space, the comprehension of their meaning, and the projection of their status in the near future”}. In other words, the ability to generate and retain good situational awareness is essential in frontline emergency handling operations.

Following the terrorist attacks in Oslo and Utøya on 22. Juli 2011, the combined emergency response was thoroughly analyzed by the 22. July commission. The analysis report \cite{nou_2012_14_rapport_2012} points out that the combined human and digital capabilities of the cooperating agencies for gathering, sharing, processing, and applying information were not scaled for handling the complex aftermath of the attack. The Police received particularly harsh critique for not satisfactorily performing the role as main information broker, mainly due to their operations central being unable to properly process and distribute available information. This prevented the agencies from retaining combined actionable situational awareness (SA). More specifically, the breakdown in SA processes led to information on vehicle identification not being available for the right actors at the right time, possibly delaying both identification and detainment of the terrorist \cite{nou_2012_14_rapport_2012}.
 
Although describing an obsolete scenario, the recounting above effectively illustrates the time-sensitive nature of SA within complex environments and the importance of effective information handling for enabling timely reactions to events. Manually retaining actionable, federated SA can be highly resource intensive, especially in complex situations with many independent actors.

During the of 30. December Gjerdrum landslide 2020, early mobilization of a large coordination management team and through knowledge of each other’s internal processes enabled the rescue agencies to gain shared SA and to rapidly deploy and coordinate resources. At most, the combined emergency effort involved more than 1000 simultaneous rescuers from multiple civilian and government agencies. 

Although regarded as a successful rescue operation, the response analysis report created by Joint Rescue Coordination Centre (HRS) suggests that close proximity to abundant material and human resources were central in the operation’s success \cite{halvorsen_evalueringsrapport_2021}. Similar emergency operations in areas with access to less resources will probably be less successful, indicating that new digitalized tools and methodologies should be developed to reduce reliance on processes, possibly increasing the chances of success in future rescue scenarios. 

\section{Gathering and distributing information}
Generating and retaining actionable frontline SA requires access to efficient tools for information handling. The emergence of smart cities and abundant sensor technology have led to the implementation of large-scale sensor systems like closed circuit television (CCTV), which has been shown to have positive impacts on the general emergency readiness \cite{thomas_internationalisation_2022}. Secondary implementations of smart city sensor technology can take advantage of the latest artificial intelligence (AI) developments, resulting in capabilities like gunshot triangulation or movement-tracking of individuals through series of camera observations \cite{arai_smart_2020}. Although singularly effective, these systems are often implemented as self-contained information silos with mostly manual interaction options. This greatly reduces timeliness of real-time SA processes and exacerbates information handling issues \cite{thomas_internationalisation_2022}.
 
Unmanned aerial vehicles (UAVs), colloquially called drones, can be highly effective in establishing situational overview or generating real-time information. Paradoxically, current UAV implementations lack automated control options and rely fully on specially trained human operators for piloting and information gathering. This makes it hard to fully utilize the capabilities of drones and renders them an expensive commodity rather than a ubiquitous resource \cite{arai_smart_2020}. This became evident during the Gjerdrum incident, where multiple aerial drones were used for monitoring the disaster zone and locating survivors. As the drones were manually controlled, were sharing the airspace with rescue helicopters, and had no automatic aerial beacons, they had to follow the same strict flight rules at the helicopters, greatly reducing their effectiveness \cite{halvorsen_evalueringsrapport_2021}.
