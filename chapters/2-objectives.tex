\chapter{Objectives}
\section {Research problems}	
\textbf{Main research problem:} There is a gap between the need for increasing real-time digital SA capabilities and the current abilities to efficiently generate and retain shared actionable SA. This indicates that new methodologies and paradigms for strengthening information handling processes are needed.

\textbf{Sub-problem 1:} There is a need for more interconnected systems. Current information handling systems are solving contained tasks within closed environments and there is no efficient digital infrastructure for communicating between information silos \cite{arai_smart_2020}. There is a need for developing infrastructure that can establish real-time information sharing between several heterogenous agents and systems, and that can support communications across multiple government and civilian organizations \cite{halvorsen_evalueringsrapport_2021}. Implementing usable digital SA approaches within this environment will necessitate real-time sharing of vast amounts of information, which will lead to exponential increases in information handling complexity.  

\textbf{Sub-problem 2:} There is a need for more intelligent systems. Generating federated SA within the described context will demand support for vast amounts of information sources with multiple information modalities. This will increase information handling complexity beyond human capabilities and high-level automation will be needed for parsing and distributing information \cite{arai_smart_2020}. There is a need for defining how high-level automation and artificial intelligence (AI) should be integrated in the described context, and what parts of the information handling processes could be entrusted to full autonomous digital entities.

\textbf{Sub-problem 3:} There is a need for better cooperation between humans and digital systems. The inclusion of intelligent and autonomous non-human actors in information handling environments will necessitate novel approaches to how technology interacts with humans \cite{burcham_comprehensive_nodate}. There is a need for developing methodologies that can ensure reliable information interchanges and shared SA between human and digital system actors. 

\section {Research goals}
The \textbf{main goal} of this project is to define and develop a novel and deployable framework for enabling federated AI supported SA in frontline police and rescue operations through increasing automation and strengthening information logistics capabilities. This will be approached through the following sub-goals: 

\textbf{Sub-Goal 1:} Design framework for distributed information handling that can support federated cyber-physical SA. Current SA infrastructure consists of isolated information handling systems that are interconnected through manual and often verbal communication channels. One important goal of this project will be to define and develop the best approaches for digitalizing and automating the information distribution processes, thus creating the backbone for shared digitalized SA. A viable starting point for this research will be to investigate on-demand ad-hoc networks and city-wide interconnectedness as described by Kashef et al \cite{kashef_smart_2021}. The research should build on findings from Aksit et al \cite{aksit_data_2023} on information gathering in complex environments and on findings from Munir et al \cite{munir_artificial_2021} on distributed fog computing solutions to AI and sensor fusion. 

\textbf{Sub-Goal 2:} Develop methodologies for utilizing autonomy, AI, machine learning in SA systems. The inclusion of AI and autonomy will be central for enabling real-time shared SA in frontline operations. If implemented correctly, AI agents could assist in all steps of the SA process, both empowering and alleviating strain on human systems operators \cite{cruise_cyber-physical_2018}. This project will investigate the possibilities and limitations of advanced systems automation in the described setting and develop methodologies for how these technologies should be deployed for greatest effect. The basis for this work will be the principles behind “cyber-physical command-guided swarm” as described by Cruise \cite{cruise_cyber-physical_2018} where layered network technology and AI are utilized for enabling scalability and autonomy in large, interconnected drone swarms, and the concepts of highly autonomous heterogenous drones described by Adams et al \cite{adams_can_2023}. 

\textbf{Sub-Goal 3:} Define approaches and methodologies for sharing SA between humans and autonomous AI actors. One goal of this project will be to evaluate and define where and how AI should be utilized within the described setting and how AI-supported big-data systems could be made more human-centric. The starting point for this research will be to investigate human-AI interactions in civilian environments, such as self-driving cars and large language models, and define how these technologies can be extended and applied to the described police and emergency context. This goal will be approached through the “cyber-physical command-guided swarm” described by Cruise \cite{cruise_cyber-physical_2018} as it describes both possible roles and usages for AI within real-time environments, as well as Burcham’s \cite{burcham_comprehensive_nodate} review on possible approaches to interaction patterns between military personnel and autonomous information systems.
