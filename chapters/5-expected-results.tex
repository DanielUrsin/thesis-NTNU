\chapter{Expected Results}

The main problem addressed by this research project is the lack of digital solutions for generating and retaining shared situational awareness for frontline police and rescue operations. Deploying real-time SA solutions within this setting is expected to increase the operational effectiveness and to create safer and more controllable situation environments. 

More specifically, analyzing the existing information handling processes is expected to compile central knowledge on the specific information needs within the described setting. On top of this, the qualitative research on possible AI/ML implementations for bolstering human capabilities and the exploration of viable human-machine interfacing techniques that can better integrate humans with SA systems are both expected to generate knowledge on how SA and information handling systems should interact with the physical environment. 

In combination, these results will lay the groundwork for investigating and defining feasible methods and approaches for implementing digitalized SA processes in the described context with the current level of technology. The results and findings from this project will have extensive application potential, both within civilian and defense contexts. This will help to further establish the necessary knowledge environment for increasing Kongsberg’s AI and ML efforts. Implementations will help to bridge the information gap between civilian and government emergency response organizations and strengthen national total-defense capabilities. 
