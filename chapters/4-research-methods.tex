\chapter{Research Methods}

This project will use applied research methods to develop digital solutions for generating and retaining shared situational awareness for frontline police and rescue operations. This will be approached through the goals specified in section 2.2, where the individual sub-goals are designed to create in-depth qualitative understanding of separate research areas and domains. The results of the sub-goals will lay the groundwork for exploring and determining viable approaches to the main project goal.

Sub-goal 3 will be reached through a state-of-the-art literature review on real-time shared SA approaches with focus on emergency and military settings, as well as content analysis on the working methods and real-time information handling routines within emergency response.

Sub-goal 2 will be reached through qualitative research on AI/ML usage and their potential roles in shared SA systems. This will be accomplished by reviewing current solutions and evaluate potential AI/ML usage in the extended context established through sub-goal 3. 

Sub-goal 1 will identify and develop viable methodologies through experimentation guided by findings form sub-goals 3 and 2, thus fulfilling the main project goal. Methodologies and potential frameworks developed through sub-goal 1 will be implemented and tested in an appropriate test environment. 

\section{Data management and sharing}
The research project intends compile and utilize data on the information handling processes within the police, armed forces, and emergency response. Some of this data may not be part of the public domain and could require that special measures for storage and usage are implemented. A data management plan will be created for ensuring safe data handling. Additionally, this project is partially founded by Forskningsrådet and is required to implement a data management plan. The data management plan will also evaluate if project data can be shared as “open-source” and what data licenses should be considered. This must be done for each individual dataset. Code and specific solutions developed by the project should also be individually evaluated before it can be shared publicly. Sharing of data and code must follow practices established by KDA. 
