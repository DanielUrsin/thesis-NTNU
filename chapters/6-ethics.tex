\chapter{Ethics}


\hspace{\parindent} \textbf{Security and civil liberties:}
Introducing automation and federated SA in security and Police contexts could have negative effects on civil liberties. Naturally, there will always be a trade-off between security and privacy in technological societies, and comprehensive ethical evaluations must be conducted before live testing or deployment of mentioned technologies could commence.

\textbf{Transparency in complex systems:}
Complex systems that employ autonomous elements will inherently be less transparent than more linear systems. Transparency in AI processes and other autonomous system elements is important for monitoring and understanding decision processes. Acceptable levels of transparency must be defined before systems that incorporate automation and federated SA can be deployed or tested in the public domain. Methodologies and potential frameworks resulting from this project should incorporate methods and functionality for supporting transparent execution and decision-making processes. 

\textbf{Decision making and accountability:}
Automated decision making is a major ethical concern in autonomous and automated systems, especially when people are involved. As well as transparency, automated systems that makes decisions in the public domain should be configured to support human-in-the-loop decision making chains. This raises issues on decision ownership across multiple stakeholders and institutions. There is a need for properly defining data and information ownership, as well as ownership or jurisdiction of the decisions made within automated SA systems. Methodologies and potential frameworks resulting from this project must consider these issues and facilitate both information and decision ownership.
